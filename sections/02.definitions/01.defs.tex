Let $\mathcal{A} = \{\alpha_i: \alpha_i \in \mathbb{R}\}$,
$i \in \texttt{I} = \langle 0,1,\dots,n-1 \rangle$, denote a set of $n$ elements,
$\langle\cdot\rangle$ denote an ordered set, $\mathcal{A}_{\uparrow}$ the
set $\mathcal{A}$ ordered in ascending order, the bracket notation
$\mathcal{A}[\texttt{I}] = \mathcal{A}$ denote indexing, and notation
$\mathcal{A}_{k:l}$, $0 \leq k \leq l$, denote limited indexing:
$\mathcal{A}_{k:l}= \{\mathcal{A}[k], \mathcal{A}[k+1], \dots, \mathcal{A}[l]\}$.



%%%%%%%%%%%%%%%%%%%%%%%%%%%%%%%%%%%%%%%%%%%%%%%%%%%%%%%%%%%%%%%%%%%%%%%%%%%%%%%%
\begin{definition}
  \label{def:definition_1} \textit{Range scan captured from a 2D LIDAR
  sensor}.---A conventional 2D LIDAR sensor provides a finite number of
  ranges, i.e. distances to objects within its range, on a horizontal
  cross-section of its environment, at regular angular and temporal intervals,
  over a defined angular range \cite{Cooper2018c}. A range scan $\mathcal{S}$,
  consisting of $N_s$ rays over an angular range $\lambda$, is an ordered map
  $\mathcal{S} : \Theta \rightarrow \mathbb{R}_{\geq 0}$,
  $\Theta = \{\theta_n \in [-\frac{\lambda}{2}, +\frac{\lambda}{2}) : \theta_n = -\frac{\lambda}{2}
  + \lambda \frac{n}{N_s}$, $n = 0,1,\dots, N_s$$-$$1$$\}$. Angles $\theta_n$
  are expressed relative to the sensor's heading, in the sensor's frame of
  reference.
\end{definition}

%%%%%%%%%%%%%%%%%%%%%%%%%%%%%%%%%%%%%%%%%%%%%%%%%%%%%%%%%%%%%%%%%%%%%%%%%%%%%%%%
\begin{definition}
  \label{def:definition_2} \textit{Map-scan}.---A map-scan is a virtual scan
  that encapsulates the same pieces of information as a scan derived from a
  physical sensor. Only their underlying operating principle is different due
  to the fact the map-scan refers to distances to the boundaries of a
  point-set, referred to as the map, rather than within a real environment. A
  map-scan $\mathcal{S}_V^{\bm{M}}(\hat{\bm{p}})$ is derived by means of
  locating intersections of rays emanating from the estimate of the sensor's
  pose estimate $\hat{\bm{p}}$ and the boundaries of the map $\bm{M}$.
\end{definition}

%%%%%%%%%%%%%%%%%%%%%%%%%%%%%%%%%%%%%%%%%%%%%%%%%%%%%%%%%%%%%%%%%%%%%%%%%%%%%%%%
\begin{definition}
  \label{def:definition_3} \textit{CAER as metric}.---Let
  $\mathcal{S}_p$ and $\mathcal{S}_q$ be two range scans, equal in angular
  range $\lambda$ and size $N_s$. The value of the Cumulative Absolute Error
  per Ray (CAER) metric $\psi \in \mathbb{R}_{\geq 0}$ between $\mathcal{S}_p$
  and $\mathcal{S}_q$ is given by
  \begin{align}
    \psi(\mathcal{S}_p,\mathcal{S}_q) \triangleq \sum\limits_{n=0}^{N_s-1} \Big| \mathcal{S}_p[n]-\mathcal{S}_q[n]\Big| \label{eq:caer}
  \end{align}
\end{definition}

%%%%%%%%%%%%%%%%%%%%%%%%%%%%%%%%%%%%%%%%%%%%%%%%%%%%%%%%%%%%%%%%%%%%%%%%%%%%%%%%
\begin{definition}
  \label{def:definition_4} \textit{CAER as field}.---A $\psi$-field on map
  $\bm{M}$
  $f_{\psi}^{\bm{M}} : \mathbb{R}^2 \times [-\pi, +\pi) \rightarrow \mathbb{R}_{\geq 0}$
  is a mappping of 3D pose configurations
  $\hat{\bm{p}}(\hat{x},\hat{y},\hat{\theta})$ to CAER values (def.
  \ref{def:definition_3}) such that if
  $\psi(\mathcal{S}_R,\mathcal{S}_V^{\bm{M}}(\hat{\bm{p}})) = c$ then
  $f_{\psi}^{\bm{M}}(\hat{\bm{p}}) = c$. In other words a CAER field is
  produced by computing the value of the CAER metric between a range scan
  $\mathcal{S}_R$ (def. \ref{def:definition_1}) and a map-scan
  $\mathcal{S}_V^{\bm{M}}(\hat{\bm{p}})$ captured from pose configuration
  $\hat{\bm{p}}$ within map $\bm{M}$ (def. \ref{def:definition_2}).
\end{definition}

%%%%%%%%%%%%%%%%%%%%%%%%%%%%%%%%%%%%%%%%%%%%%%%%%%%%%%%%%%%%%%%%%%%%%%%%%%%%%%%%
\begin{definition}
  \label{def:definition_5} \textit{Rank field}.---Let $f_{\psi}^{\bm{M}}$ be
  a $\psi$-field on map $\bm{M}$ and $\mathcal{P} = \{\hat{\bm{p}}_i\}$,
  $i \in \texttt{I} = \langle 0,1,\dots,|\mathcal{P}|-1 \rangle$, be a set of 3D pose
  configurations within map $\bm{M}$, such that
  $f_{\psi}^{\bm{M}}(\mathcal{P}) = \Psi$. Let $\texttt{I}^{\ast}$
  be the set of indices such that $\Psi[\texttt{I}^{\ast}] = \Psi_\uparrow$.
  A $\texttt{r}$-field on map $\bm{M}$
  $f_{\texttt{r}}^{\bm{M}} : \mathbb{R}^2 \times [-\pi, +\pi) \rightarrow \mathbb{Z}_{\geq 0}$
  is a mapping of 3D pose configurations $\mathcal{P}$ to non-negative integers
  such that if $f_{\psi}^{\bm{M}}(\mathcal{P}) = f_{\psi}^{\bm{M}}(\mathcal{P}[\texttt{I}]) = \Psi$ then
  $f_{\texttt{r}}^{\bm{M}}(\mathcal{P}[\texttt{I}^\ast]) = \texttt{I}$.
  %(equivalently:
  %$f_{\texttt{r}}^{\bm{M}}(\mathcal{P}) = \texttt{I}^\ast$)
  In other words a rank field maps the elements of pose estimate set $\{\hat{\bm{p}}_i\}$
  to the ranks $\texttt{I}^\ast$ of their corresponding CAER values in
  hierarchy $\Psi_\uparrow$.
\end{definition}

%%%%%%%%%%%%%%%%%%%%%%%%%%%%%%%%%%%%%%%%%%%%%%%%%%%%%%%%%%%%%%%%%%%%%%%%%%%%%%%%
\begin{definition}
  \label{def:definition_6} \textit{Field densities}.---The locational and
  angular density, $d_l$ and $d_{\alpha}$ respectively, of a
  $\psi$- or $r$-field express, correspondingly, the number of pose estimates
  per unit area of space and per angular unit.
\end{definition}

%%%%%%%%%%%%%%%%%%%%%%%%%%%%%%%%%%%%%%%%%%%%%%%%%%%%%%%%%%%%%%%%%%%%%%%%%%%%%%%%
\begin{definition}
  \label{def:definition_7} \textit{Admissibility of solution}.---A pose estimate
  $\hat{\bm{p}}(\hat{x},\hat{y},\hat{\theta}) \in \mathbb{R}^2 \times [-\pi,+\pi)$
  may be deemed an admissible solution to Problem \ref{prob:the_problem} iff
  $\|\bm{l}-\hat{\bm{l}}\|_2 \leq \delta_{\bm{l}}$ and
  $|\theta-\hat{\theta}| \leq \delta_\theta$, where
  $\delta_{\bm{l}}, \delta_\theta \in \mathbb{R}_{> 0}$ and
  $\delta^2 = \delta_{\bm{l}}^2 + \delta_{\theta}^2$.
\end{definition}
