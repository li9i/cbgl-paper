This article has presented a single-shot Monte Carlo approach to the solution
of the passive version of the global localisation problem with the use of a 2D
LIDAR sensor, titled CBGL. CBGL allows for the fast estimation of the sensor's
pose within a metric map by first dispersing hypotheses in it and then
leveraging the proportionality of values of the Cumulative Absolute Error per
Ray (CAER) metric to the pose errors of the hypotheses, for estimates in a
neighbourhood of the sensor's pose. CBGL was evaluated in various real and
simulated conditions and environments; it was found to be superior to Monte
Carlo and feature-based approaches in terms of number of inlier pose estimates
and execution time. Future steps will aim at (a) the extension of the CAER
metric for the use with 3D LIDAR sensors in service to a solution of the
problem of their localisation in 6DoF, and (b) making CBGL more robust
by considering the statistics of clusters in case of diverse estimates in its
$\mathcal{H}_2$ set. The C++ ROS code of the proposed method is available at
\url{https://github.com/li9i/cbgl}.
