Navigation of a mobile robot is conditioned on the knowledge of its pose. In
observer-based localisation configurations its initial pose may not be knowable
in advance, leading to the need of its estimation. The Monte Carlo class of
solutions disperse hypotheses in the environment's map, which gradually
converge on an estimate. These methods are robust against noise but require
motion and time, which may be economised on. Feature-based solutions require
minimal estimation time but assume environmental structure, may be sensitive to
noise, and require preprocessing and tuning. The method proposed in this
article retains the positive qualities of the two main approaches to global
localisation and avoids their pitfalls: it requires a single measurement from a
2D LIDAR sensor, assumes no motion, environmental structure, or parameter
tuning in a per-environment or sensor basis, and is robust against noise.  A
large number of tests, in real and simulated conditions, involving disparate
environments and sensor properties, illustrate that the proposed method
outperforms state-of-the-art methods of both classes in terms of pose discovery
rate and execution time. The source code is available for download.
