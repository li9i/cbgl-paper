The literature concerning the solution to the problem of global localisation
with the use of a 2D LIDAR sensor is rich; a brief overview is given below.

In broad terms global localisation approaches may be divided into two
categories: (a) approaches that operate in feature space, that is, methods that
extract features from measurements and the map and then establishing
correspondences between them, and (b) approaches that directly exploit only raw
measurements. In the latter category a number of global localisation methods
solve the problem in an iterative Bayesian Monte Carlo fashion, i.e. by
dispersing hypotheses within the map and updating the belief of the robot's
pose by incorporating new measurements as it moves until estimate convergence
\cite{mcl,Wang2018d,Yilmaz2019a,gmcl,Chen2021a}. However, the requirement of
motion (a) may give rise to safety concerns (the robot may not even be
visible), and (b) increases estimation time. The Monte Carlo method proposed in
this article operates directly in measurement space as well but, in contrast,
it does not require motion or more than one measurement. As a result CBGL is a
single-shot global localisation approach that is executable in less time
compared to previous Monte Carlo approaches.

Research on the former category has been more extensive due to the richness,
appropriateness, adaptability, and efficacy of methods originated in the
computer vision field. Relevant methods mainly perform detection of key-points
in a measurement, followed by the calculation of a distinctive signature,
which is then matched to similarly- and pre-computed place-signatures
\cite{Kallasi2016a,Usman2019,Wang2021b,Meng2021,Hendrikx2021,An2022,Nielsen2023}.
In principle, however, unstructured environments cannot be relied upon for the
existence of features, due to their complete absence or their sparse and
fortuitous distribution (although Deep Neural Network approaches have
demonstrated increased performance in place recognition with the use of 3D
LIDARs \cite{Xu2021,Yin2022,Komorowski2022}). Structured environments on the
other hand manifest different features depending on the particularities of the
environment. In any case, features may be present but not in a sufficiently
undisturbed state due to sensor noise or map-to-environment mismatch.

The motivation of the proposed method originates from seeking to achieve a
greater degree of universality, reliability, and portability across multiple
and disparate environments, by not making a priori assumptions on environment
structure, and by aiming to minimise requirements on resources (number and
types of sensors; number of measurements; time). The proposed method is most
akin to the two tested methods in \cite{Filotheou2022g}: all three are
single-shot 2D LIDAR-based Monte Carlo approaches, but CBGL computes a measure
of the alignment of (a) the measurement to (b) the virtual scan captured from
each hypothesis \textit{before} scan--to--map-scan matching a subset of the
most-aligned poses to the measurement. As a consequence it is able to process
more hypotheses in less time, resulting in (a) an improvement over the number
of correctly estimated locations and, most starkly, (b) execution time, being
able to compete against (traditionally faster) feature-based approaches.

A recent and comprehensive survey on global localisation may be found in
\cite{gl_survey_cn}.
