The literature concerning the solution to the problem of global localisation
with the use of a 2D LIDAR sensor is rich. A recent and comprehensive survey on
global localisation may be found in \cite{gl_survey_cn}, while a brief overview
is given below.

In broad terms global localisation approaches may be divided into two
categories: (a) approaches that operate in feature space, that is, methods that
extract features from measurements and the map and establish correspondences
between them, and (b) approaches that directly exploit only raw measurements.
In the latter category a number of methods solve the problem in an iterative
Bayesian Monte Carlo fashion, i.e. by dispersing hypotheses within the map and
updating the belief of the robot's pose by incorporating new measurements as it
moves until estimate convergence
\cite{mcl,Wang2018d,Yilmaz2019a,gmcl,Chen2021a}. However, the requirement of
motion (a) may give rise to safety concerns (the robot may not even be
visible), and (b) increases estimation time. The Monte Carlo method proposed in
this article operates directly in measurement space as well but, in contrast,
is not iterative, and does not require motion or more than one measurement. As
a result CBGL is a single-shot global localisation approach that is able to
process more hypotheses in less time, resulting in an improvement in the number
of correctly estimated locations, and making it able to compete against
(traditionally faster) feature-based approaches in terms of execution time.

Research on the feature-based approaches has been more extensive due to the
richness, adaptability, and efficacy of methods originated in the computer
vision field, and their low execution time. Relevant methods mainly perform
detection of key-points in a measurement, followed by the calculation of a
distinctive signature, which is then matched to similarly- and pre-computed
place-signatures
\cite{Kallasi2016a,als_eth,Usman2019,Wang2021b,Meng2021,Hendrikx2021,An2022,Nielsen2023}.
In principle, however, unstructured environments cannot be relied upon for the
existence of features due to their absence or their sparse and fortuitous
distribution (although Deep Neural Network approaches have demonstrated
increased performance in place recognition with the use of 3D LIDARs
\cite{Xu2021,Yin2022,Komorowski2022}). Structured environments, on the other
hand, manifest different features depending on the particularities of the
environment, where features may be present but not in a sufficiently
undisturbed state by sensor noise or map-to-environment mismatch. Furthermore,
feature-based methods require the tuning of parameters in a per-environment
basis, which hinders the range and degree of their applicability and efficacy.
In contrast CBGL (a) makes no assumptions on- and exploits no environmental
structure and (b) uses three parameters, whose setting is optional and
intuitive.

In sum the proposed method retains most of the positive qualities of the two
main approaches to global localisation and avoids their pitfalls: it uses
multiple hypotheses for robustness against uncertainty, assumes no motion,
environmental structure, or parameter tuning in a per-environment or sensor
basis, and is robust against noise.
The motivation of CBGL originates from seeking to achieve a
greater degree of universality, reliability, and portability across multiple
and disparate environments, by aiming to economise on the use of resources
(environmental assumptions; number, type, and cost of sensors; number of
measurements; time). The proposed method is most akin to the two tested
methods in \cite{Filotheou2022g}: all three are single-shot 2D LIDAR-based
Monte Carlo approaches, but CBGL computes a measure of the alignment of the
measurement to the virtual scan captured from each hypothesis
\textit{before} scan--to--map-scan matching, which occurs subsequently and only
for a small subset of the most-aligned virtual scans to the measurement. This,
along with the low computational complexity of the alignment measure that CBGL
utilises, greatly diminish its execution time. So far this alignment measure
(Def.  \ref{def:definition_3}) has only been tested against the context of
\texttt{sm2} in pose tracking, i.e. for pose hypotheses in a neighbourhood of
the sensor's true pose%, and its computational complexity is low
\cite{Filotheou2022f,Filotheou2023a}.
