% Colors:
% Blue-ish: 8FC2FF
% Yellow-ish: FFE599
% Red-ish: FF9999
% Brown-ish: E5C07B
% Blue/Green-ish: 397880

% Remove eventually
%\usepackage{showframe}
%\usepackage{lipsum}

\setcounter{secnumdepth}{3}
\usepackage{listings}
\usepackage{color}
\usepackage[hidelinks]{hyperref}
\let\proof\relax
\let\endproof\relax
\usepackage{amsmath}
\usepackage{amssymb}
\usepackage{amsthm}
\usepackage{subfig}
\usepackage{balance}
\usepackage{stfloats}
\usepackage{graphicx}
\usepackage{epstopdf}
\usepackage{bm}
\usepackage{booktabs}
\usepackage{array}
\usepackage{stfloats}
\usepackage{pdflscape}
\usepackage{pgfplots}
\usepackage{algorithm}
\usepackage{algorithmic}
\usepackage{setspace}
\usepackage{tikz}

%\usepackage[noadjust]{cite}
\usepackage{cite}

\renewcommand{\algorithmicrequire}{\textbf{Input:}}
\renewcommand{\algorithmicensure}{\textbf{Output:}}
\renewcommand\thealgorithm{\Roman{algorithm}}

% Italics for theorems
\theoremstyle{plain}
\newtheorem{theorem}{Theorem}
\renewcommand\thetheorem{\Roman{theorem}}

% Normal for all the rest
\newtheorem{lemma}{Lemma}
\renewcommand\thelemma{\Roman{lemma}}
\newtheorem{corollary}{Corollary}
\renewcommand\thecorollary{\Roman{corollary}}
\theoremstyle{definition}
\newtheorem{problem}{Problem}
\renewcommand\theproblem{\Roman{problem}}

% https://tex.stackexchange.com/questions/53978/custom-theorem-numbering
\newtheorem{innercustomprb}{Problem}
\newenvironment{customprb}[1]
  {\renewcommand\theinnercustomprb{#1}\innercustomprb}
  {\endinnercustomprb}

\newtheorem{innercustomhpt}{Hypothesis}
\newenvironment{customhpt}[1]
  {\renewcommand\theinnercustomhpt{#1}\innercustomhpt}
  {\endinnercustomhpt}

\newtheorem{innercustomobs}{Observation}
\newenvironment{customobs}[1]
  {\renewcommand\theinnercustomobs{#1}\innercustomobs}
  {\endinnercustomobs}

\newtheorem{innercustomcnj}{Conjecture}
\newenvironment{customcnj}[1]
  {\renewcommand\theinnercustomcnj{#1}\innercustomcnj}
  {\endinnercustomcnj}

\newtheorem{definition}{Definition}
\renewcommand\thedefinition{\Roman{definition}}
\newtheorem{remark}{Remark}
\renewcommand\theremark{\Roman{remark}}
\newtheorem{proposition}{Proposition}
\renewcommand\theproposition{\Roman{proposition}}
\newtheorem{objective}{Objective}
\renewcommand\theobjective{\Roman{objective}}
\newtheorem{operation}{Operation}
\renewcommand\theoperation{\Roman{operation}}

\renewcommand{\qedsymbol}{$\blacksquare$}

\definecolor{mygray}{rgb}{0.5,0.5,0.5}
\definecolor{keyword}{rgb}{0.5,0.5,0.5}
\definecolor{greenCode}{rgb}{0, 0.6, 0}

\lstdefinelanguage{HTTP}{
  keywords={GET},
  ndkeywords={PUT},
  comment=[s]{PO}{T},
  morecomment=[s]{D}{LETE}
}

\lstdefinestyle{customc}{
  belowcaptionskip=1\baselineskip,
  language={HTTP},
  breaklines=true,
  frame=tb,
  captionpos=b,
  keywordstyle=\bfseries\color{greenCode},
  ndkeywordstyle=\bfseries\color{red},
  commentstyle=\bfseries\color{magenta},
  stringstyle=\bfseries\color{black},
  xleftmargin={0.75cm},
  showstringspaces=false,
  basicstyle=\footnotesize\ttfamily,
  numbers=left,
  numberstyle=\small\color{black},
}

\lstset{escapechar=@,style=customc}

%\ifCLASSINFOpdf
   %\usepackage[pdftex]{graphicx}
%\else
%\fi

\usepackage{flushend} % Equalize last page
\usepackage[colorinlistoftodos]{todonotes}

\usepackage{framed}

% Manipulate vspace for algorithm
\makeatletter
\newcommand\fs@spaceruled{\def\@fs@cfont{\bfseries}\let\@fs@capt\floatc@ruled
  \def\@fs@pre{\vspace{0.5\baselineskip}\hrule height.8pt depth0pt \kern2pt}%
  \def\@fs@post{\kern2pt\hrule\relax}%
  \def\@fs@mid{\kern2pt\hrule\kern2pt}%
  \let\@fs@iftopcapt\iftrue}
\makeatother


% Highlight algorithm parts with color
\newcommand*{\tikzmk}[1]{\tikz[remember picture,overlay,] \node (#1) {};\ignorespaces}
%define a boxing command, argument = colour of box
\newcommand{\boxit}[1]{\tikz[remember picture,overlay]{\node[yshift=3pt,fill=#1,opacity=.25,fit={($(A)+(-0.038\linewidth,.3\baselineskip)$)($(B)+(.9\linewidth,-.3\baselineskip)$)}] {};}\ignorespaces}
