%%%%%%%%%%%%%%%%%%%%%%%%%%%%%%%%%%%%%%%%%%%%%%%%%%%%%%%%%%%%%%%%%%%%%%%%%%%%%%%%
\begin{definition}
  \label{def:definition_1}
  A conventional 2D LIDAR sensor provides a finite number of ranges,
  i.e. distances to objects within its range, on a horizontal cross-section of
  its environment, at regular angular and temporal intervals, over a defined
  angular range \cite{Cooper2018b}. A range scan $\mathcal{S}$, consisting
  of $N_s$ rays over an angular range $\lambda$, is an ordered map
  $\mathcal{S} : \Theta \rightarrow \mathbb{R}_{\geq 0}$, $\Theta =
  \{\theta_n \in [-\frac{\lambda}{2}, +\frac{\lambda}{2}) : \theta_n =
  -\frac{\lambda}{2} + \lambda \frac{n}{N_s}$, $n = 0,1,\dots, N_s$$-$$1$$\}$.
  Angles $\theta_n$ are expressed relative to the sensor's heading, in the
  sensor's frame of reference.
\end{definition}

%%%%%%%%%%%%%%%%%%%%%%%%%%%%%%%%%%%%%%%%%%%%%%%%%%%%%%%%%%%%%%%%%%%%%%%%%%%%%%%%
\begin{definition}
  \label{def:definition_2}
  A map-scan is a virtual scan that encapsulates the same pieces of information
  as a scan derived from a physical sensor. Only their underlying operating
  principle is different due to the fact the map-scan refers to distances to
  the boundaries of a point-set, referred to as the map, rather than within a
  real environment. A map-scan $\mathcal{S}_V^{\bm{M}}(\hat{\bm{p}})$ is
  derived by means of locating intersections of rays emanating from the
  estimate of the sensor's pose estimate $\hat{\bm{p}}$ and the boundaries of
  the map $\bm{M}$.
\end{definition}

%%%%%%%%%%%%%%%%%%%%%%%%%%%%%%%%%%%%%%%%%%%%%%%%%%%%%%%%%%%%%%%%%%%%%%%%%%%%%%%%
\begin{definition}
  \label{def:definition_3}
  Let $\mathcal{S}_p$ and $\mathcal{S}_q$ be two range scans, equal in angular
  range $\lambda$ and size $N_s$. The value of the Cumulative Absolute Error
  per Ray (CAER) metric $\psi \in \mathbb{R}_{\geq 0}$ between $\mathcal{S}_p$
  and $\mathcal{S}_q$ is given by
  \begin{align}
    \psi(\mathcal{S}_p,\mathcal{S}_q) \triangleq \sum\limits_{n=0}^{N_s-1} \Big| \mathcal{S}_p[n]-\mathcal{S}_q[n]\Big| \nonumber
  \end{align}
\end{definition}

%%%%%%%%%%%%%%%%%%%%%%%%%%%%%%%%%%%%%%%%%%%%%%%%%%%%%%%%%%%%%%%%%%%%%%%%%%%%%%%%
\begin{customprb}{P}
  \label{prob:the_problem}
  Let the unknown pose of an immobile 2D range sensor whose angular range is
  $\lambda$ be $\bm{p}(x,y,\theta)$ with respect to the reference frame of map
  $\bm{M}$. Let the range sensor measure range scan $\mathcal{S}_R$. The
  objective is the estimation of $\bm{p}$ given $\bm{M}$, $\lambda$, and
  $\mathcal{S}_R$.
\end{customprb}
